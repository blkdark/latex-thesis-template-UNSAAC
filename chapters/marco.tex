	\chapter{MARCO TEÓRICO CONCEPTUAL}	
		\section{Bases Teóricas}
	
			\subsection{Teoría ...}	
				\subsubsection{Juegos...}
				
				\begin{quote}
					Por una parte, los modelos de la teoría de juegos cooperativos están definidos por un conjunto de agentes y una función característica de beneficio o coste. Por otra parte, los modelos de la teoría de juegos no-cooperativos están definidos por un conjunto de agentes y una estructura de información que contiene un conjunto de acciones de cada agente, estrategias, pagos y la secuencia de movimientos en la interacción del juego. \parencite[p. VI]{valencia-toledoBargainingModelsAsymmetric2017} 
				\end{quote}		
				
				\subsubsection{Forma...}
					
				Especificando los tres elementos, tenemos:
				\begin{enumerate}
					\item Los jugadores.
					\item Las estrategias para cada jugador.
					\item El bienestar o Utilidad que el jugador obtiene por la unión de las estrategias posibles del juego.
				\end{enumerate}	
				
					\paragraph{Equilibrio de Nash}
					
					formalmente podríamos decir:
					
					\textit{Definición:} En un conjunto de estrategias combinadas \((s^N_1,...,s^N_n)\) es un equilibrio de Nash si para todo jugador \(i\) en \(N\) se cumple que
					\[u_i(s^N_1,...,s^N_i-1,s^N_i,s^N_i+1,...,s^N_n)\geqq u_i(s^N_1,...,s^N_i-1,s_i,s^N_i+1,...,s^N_n)\]
			
			\subsection{Colusión}
			
			Uno de los elementos centrales en las investigaciones de colusión es la dificultad de probar su existencia, ya que usualmente no se cuenta con evidencia directa como documentos escritos, grabaciones o confesiones de parte. En estos casos, las autoridades de competencia tienden a utilizar evidencia circunstancial, gran parte de la cual está conformada por evidencia económica: pruebas de la conducta de la empresa, la estructura del mercado y la existencia de prácticas que faciliten el mantenimiento de acuerdos colusorios.
		
					\paragraph{Colusión Explicita}
					
					La colusión explícita se produce cuando las empresas miembros de un oligopolio tienen comunicación y/o contacto efectivo entre ellas que les permite coordinar su comportamiento a fin de dejar de competir. En términos legales este comportamiento se califica como acuerdos o prácticas concertadas y se denomina genéricamente como carteles.
					
					\paragraph{Colusión Tácita}
					la colusión tácita supone que las empresas pueden ajustar su comportamiento al de sus rivales sin necesidad de contacto ni comunicación directa entre ellas. 
					Esto es posible debido a las particulares condiciones del mercado que les permiten detectar oportunamente la desviación del comportamiento esperado y, además, hacen creíble que aplicarán medidas de represalia frente a las desviaciones que de identifiquen. La respuesta legal frente a este comportamiento aún no se encuentra definida, existiendo planteamientos que señalan que no debería ser sancionado por ser una conducta racional y lógica en mercados con tales características.
			

\newpage
		\section{Marco Conceptual}	
		
			\subsection*{Empresa de Souvenirs}
			Agente Económico que dentro del mercado turístico se enfoca en la venta de productos recordatorios ``souvenirs'' de diversa índole, pudiendo ser productos textiles, cerámicos, orfebres, etc. 
			\subsection*{Agencias de Turismo}
			Empresa Privada que dentro del mercado turístico se encarga de la mediación, organización y realización de itinerarios para los turistas.
			\subsection*{Turista}
			Persona que se desplaza de un lugar geográfico no originario a otro con finalidad de conocer, pasear, en un lugar ajeno.
				
		\section{Antecedentes Empíricos de la Investigación}
		
		 A nivel nacional ...
		 
		