\documentclass[11pt]{beamer}
\usepackage[utf8]{inputenc}
\usepackage[T1]{fontenc}
\usepackage{lmodern}
\usepackage[spanish]{babel}
\usepackage{amsmath}
\usepackage{amsfonts}
\usepackage{amssymb}
\usepackage{graphicx}
\usetheme{Warsaw}
\usecolortheme{crane}
\useoutertheme{shadow}
\useinnertheme{rectangles}

	\author{\textbf{Blake Toscani Apaza Pérez}}
	\title[Game Theory]{\textbf{MODELO DE TEORÍA DE JUEGOS DINÁMICOS PARA EL ESTUDIO DE PRACTICAS DE COLUSION TÁCITA EN EL SECTOR TURÍSTICO DE CUSCO}}
	%\subtitle{}
	%\logo{}
	\institute{{\large UNSAAC}\\ \vspace{0.5cm}ESCUELA DE POSGRADO\\ MAESTRÍA EN MATEMÁTICA}
	\date{\today}
	%\subject{}
	%\setbeamercovered{transparent}
	%\setbeamertemplate{navigation symbols}{}
\begin{document}
%%%%%%%%%%%%%%%	
	\begin{frame}[plain]
	\maketitle
	\end{frame}
%%%%%%%%%%%%%%%
	\begin{frame}{INDICE}
	\tableofcontents
	\end{frame}
%%%%%%%%%%%%%%%%
	\begin{frame}
	\frametitle{Formulación del Problema}
	\section{Formulación del Problema}
		
		\textbf{PROBLEMA GENERAL}
		\subsection{Problema General}
		\begin{itemize}
			\item ¿Se puede expresar la colusión tacita en el pago de comisiones a agentes de  y/o agencias de turismo por parte de las empresas de souvenirs   en la planicie de Sacsayhuaman como un modelo basado en teoría de juegos dinámicos?	
		\end{itemize}
		\textbf{PROBLEMAS ESPECÍFICOS}
		\subsection{Problemas Específicos}
		\begin{itemize}
			\item ¿Como es la dinámica de pago de comisiones a agentes  y/o agencias de turismo por parte de las empresas de souvenirs   en la planicie de sacsayhuaman?
			\item ¿Como conceptualizar el pago de comisiones a agentes y/o agencias de turismo como un tipo de colusión tacita?
			\item ¿Como definir un modelo basado en teoría de juegos dinámicos?
		\end{itemize}
	\end{frame}
%%%%%%%%%%%%%%%%%%%
	\begin{frame}
	\frametitle{Formulacion del problema}
	\section{Formulacion del Problema}
		
		\textbf{OBJETIVO PRINCIPAL}
		\subsection{Objetivo Principal}
		\begin{itemize}
			\item Proponer la dinámica de pago de comisiones a agentes de turismo y/o agencias de turismo por parte de las empresas de souvenirs   en la planicie de sacsayhuaman como un modelo basado en teoría de juegos dinámicos.
		\end{itemize}
		
		\textbf{OBJETIVOS ESPECÍFICOS}
		\subsection{Objetivos Específicos}
		\begin{itemize}
			\item Describir la dinámica de pago de comisiones a agentes de turismo y/o agencias de turismo por parte de las empresas de souvenirs   en la planicie de sacsayhuaman. 
			\item Conceptualizar el pago de comisiones a agentes y/o agencias de turismo como un tipo de colusión tacita
			\item Definir un modelo basado en teoría de juegos dinámicos.
		\end{itemize}
		
	\end{frame}
%%%%%%%%%%%%%%%%%%%
	\begin{frame}{Marco Teórico}
	\section{Marco Teórico}
	
	\subsection{Teoría de Juegos}
	\textbf{TEORÍA DE JUEGOS}
	\begin{itemize}
		\item Juegos No-cooperativos
		\item Equilibrio de Nash
		\item Oligopólico de Cournot
		\item Juegos Dinámicos
	\end{itemize}
	
	
	
	\end{frame}
%%%%%%%%%%%%%%%%%%%
	\begin{frame}{Marco Teórico}
	
	
	\subsection{Colusión}	
	\textbf{COLUSION}
		\\Tipos.
		\begin{itemize}
			\item Colusion Explicita
			\item Colusion Tacita
		\end{itemize}
	
	\end{frame}
%%%%%%%%%%%%%%%%%%%
	\begin{frame}{METODOLOGÍA}
	\section{Metodología}
	
	\textbf{-Ámbito de estudio: localización política y geográfica}
	
	Geográficamente el área de estudio se encuentra situada en la explanada del conjunto arqueológico Sacsayhuaman en el distrito de Cusco, provincia de Cusco, departamento de Cusco.
	

	\textbf{-Tipo de Investigación}
	
	Por el tipo de investigación el presente estudio será una investigación aplicada ya que usando las herramientas matemáticas que da la teoría de juegos se analizará y modelizará la colusión tácita en un mercado oligopólico.
		
	\end{frame}
%%%%%%%%%%%%%%%%%%%%%%
	\begin{frame}{METODOLOGÍA}
	\textbf{-Nivel de Investigación}
	
	Por la naturaleza del trabajo de investigación, el presente  estudio es  descriptivo.
	
	\textbf{-Unidad de análisis}
	
	la unidad de Análisis son las empresas de souvenirs de la explanada del conjunto arqueológico Sacsayhuaman.
	
	\textbf{-Población de estudio}	
	
	Empresas de souvenirs de la explanada del conjunto arqueológico Sacsayhuaman.
\end{frame}
%%%%%%%%%%%%%%%%%%%%%
	\begin{frame}
		\centering Gracias...
	\end{frame}
\end{document}