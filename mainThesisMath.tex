%%%%%%%%%%%%%%%%%%%%BLKDARK%%%%%%%%%%%%%%%%%%%%%%%%%
%%%%formato  Latex para tesis de maestria UNSAAC%%%%
%%%%%%%%BlakeToscani%%%blkantro@yahoo.es%%%%%%%%%%%%

\documentclass[12pt,oneside]{book}

\usepackage{fontspec}
\usepackage{polyglossia}
	\setdefaultlanguage{spanish}

\usepackage[left=3.5cm, right=2.5cm, top=2.5cm, bottom=2.5cm]{geometry}

\setcounter{tocdepth}{4}%nivel de numeracion en tabla de contenidos
\setcounter{secnumdepth}{5} %nivel de numeracion en titulos en documento
\usepackage{amsmath}
\usepackage{amssymb}

\usepackage{lipsum}

%\usepackage[spanish,es-lcroman,es-nosectiondot,es-tabla]{babel} % lcroman se usa para enumerar con romanos en minuscula ya que no es correcto el uso de romanos mayusculas en español,  es-noindentfirst se usa para quitar la sangria en los parrafos, es-tabla cambia nombre de cuadros a tabla, es-nosectiondotquita punto a la enumaracion de capitulos y subcapitulos
\usepackage{parskip} %eliminar indentacion a todo el documento

%%%Bibliografia%%%%%%%
\usepackage[backend=biber,style=apa]{biblatex}
	\addbibresource{ThesisMath.bib}
\usepackage{csquotes}

\usepackage{hyperref} %referencias cruzadas
	\hypersetup{hidelinks}
%%%%%%%%%%%%%%%%%%%%%%

\setlength{\parskip}{1.5mm}
\usepackage{graphicx}
	\graphicspath{{Images/}}
\usepackage{mfirstuc} %para capitalizar letras
	\gMFUnocap{de}
	\gMFUnocap{y}
	\gMFUnocap{a}
	\gMFUnocap{de}%
\usepackage{pdfpages} %para incluir pdfs externos
\usepackage{lscape}

\usepackage{array,multirow,multicol,makecell} %PARA TABLAS
\usepackage{rotating}
\usepackage{float}
\usepackage{booktabs} 
\usepackage{wrapfig}
\usepackage{tabularx}
%\usepackage{xcolor}
\usepackage[titles]{tocloft}  %para reducir espacio entre capitulos en el indice
	\setlength{\cftbeforechapskip}{3pt} 
%\usepackage{tikz}
\usepackage{appendix}
\usepackage{setspace} %para interlineado en los parrafos... \doublespacing (doble espacio) \onehalfspace(un espacio) \singlespace (medio espacio) \spacing{1.5} (definir)
%\usepackage[usenames,dvipsnames]{xcolor} %para grafica de funciones
%\usepackage{tkz-fct}
	
%%%%%%%%%%%%%%%%%%%%%%%%%%%%%%%%%%%%%%%%%%%%%%%%%%%%%%%%%%%%%%%	
\title{ 	
		UNIVERSIDAD NACIONAL SAN ANTONIO ABAD DEL CUSCO \\
		{\large  ESCUELA DE POSGRADO}\\
		{\large MAESTRIA EN  MATEMATICAS}\\...\\
		\includegraphics[scale=0.19]{unsaac} \\.\\
		La Cultura de las Comisiones y su impacto en la informalidad en el sector turistico de Cusco }
	
\author{ Proyecto de tesis presentado por:\\
		Br. Juan Perez\\
		Para optar el grado academico de Maestro en Matematica\\
		ASESOR: NOMBRE\\
		CUSCO-PERU\\
		2019
		}
%%%%%%%%%%%%%%%%%%%%%%%%%%%%%%%%%%%%%%%%%%%%%%%%%%%%%%%%%%%%%%%

\begin{document}

%\renewcommand{\appendixpagename}{Anexos}
%\renewcommand{\appendixtocname}{Anexos}
%\renewcommand{\appendixname}{Anexo}
%\maketitle
%\setcounter{chapter}{1}

%%%%%%inserta titulo creado en la carpeta caratula%%%%%%%
\includepdf[pages=1]{./caratula/caratula.pdf}
\thispagestyle{empty} %para no numerarse en el indice
%%%%%%%%%%%%%%%%%%%%%%%%%%%%%%%%%%%%%%%%%%%%%%%%%%%%%%

%\spacing{1.5}  %para poner parrafo a espacio y medio el documento

\frontmatter 

%%%%%Dedicatoria%%%%% SOLO TESIS
\newpage	
\chapter*{}
\thispagestyle{empty}		% para quitar numeracion	
\addcontentsline{toc}{chapter}{Dedicatoria}
\begin{flushright}
	\vspace*{5cm} 
	\textit{A mi Papá, porque desde siempre allanó mi camino \\ hacia la gran aventura del pensamiento.\\A mi Mamá, porque pudo brindarme las herramientas necesarias\\ para lograr mis objetivos.}
\end{flushright}
%%%%%%%%%%%%%%%%%%%	

%%%%%Agradecimiento%%%%% SOLO TESIS
\newpage
\chapter*{}
\thispagestyle{empty}		% para quitar numeracion
\addcontentsline{toc}{chapter}{Agradecimiento}
\begin{flushright}
	\vspace*{5cm} 
	\textit{Agradecimientos.}
\end{flushright}
%%%%%%%%%%%%%%%%%%%	

%%%%%%%%%%%%%%%%%%% 
\spacing{1.5}  %para poner parrafo a espacio y medio el documento
	
	\cleardoublepage %necesario para enumerar correctamente tablas y figuras
	\addcontentsline{toc}{chapter}{Indice General}
	\tableofcontents
	
	\cleardoublepage %SOLO TESIS
	\addcontentsline{toc}{chapter}{Lista de Cuadros} %SOLO TESIS
	\listoftables %SOLO TESIS
	
	\cleardoublepage %SOLO TESIS
	\addcontentsline{toc}{chapter}{Lista de Figuras} %SOLO TESIS
	\listoffigures %SOLO TESIS

%%%%%%%%%%%%%%%%%%%

%%%%%Resumen%%%%% SOLO TESIS
\newpage
\chapter*{\centering \begin{normalsize} RESUMEN \end{normalsize}}
\thispagestyle{empty}		% para quitar numeracion
\addcontentsline{toc}{chapter}{Resumen}

La presente tesis de investigación tiene como objetivo determinar el problema de modelamiento del movimiento de un proyectil que se formulan mediante la existencia de las ecuaciones diferenciales fraccionarias no lineales.
Para la solución se presenta el estudio de las condiciones para la
recorrido. \\

{\bf Palabras Clave:} AME, keyword B, BKJSD, 

%%%%%Abstract%%%%% SOLO TESIS
\newpage
\chapter*{\centering \begin{normalsize} ABSTRACT \end{normalsize}}
\thispagestyle{empty}		% para quitar numeracion
\addcontentsline{toc}{chapter}{Abstract}

The objective of this research thesis is to determine the mathematicalmodels of a projectile that are formulated through the existence of nonlinear fractional differential equations for their best modeling in their path.
For the solution, the study of the conditions for the existence of a fractional differential equation is presented. The differential operator and the initial conditions are taken in the sense of Caputo. The procedure followed to achieve the objectives outlined is following the intuitive idea of Peano's theorem and existence of the upper and lower solution of an ordinary differential equation. to then extend together with the Taylor polynomial and formulate the mathematical models of the movement of a projectile.
In this thesis, with the support of Wolfran Mathematica 8 computational tools, it has been possible to show the graphics of the movement of a projectile more accurate in its journey. \\

{\bf Keywords:} AME, keyword B, BKJSD, 

%%%%%%%%%%%%%%%%%%%	

%%%%%%%%%%%%%%%%%%%	
				
\mainmatter
\spacing{1.5}  %para poner parrafo a espacio y medio el documento
		\chapter*{\centering INTRODUCCIÓN}
		\addcontentsline{toc}{chapter}{Introduccion}
	
	 %SOLO TESIS
		\chapter{Planteamiento del Problema}	
		\section{Situación Problemática}
				
		\section{Formulación del Problema}
			\subsection{Problema General}
			\begin{itemize}
				\item ¿Se puede...?	
				\item ¿Existe un ...?
			\end{itemize}
			\subsection{Problemas Específicos}
			\begin{itemize}
			 	\item ¿Como es l...?
			 	\item ¿los agentes ...?
			 	\item ¿Como ...?
			\end{itemize}
		\section{Justificación de la Investigación}
			
		\section{Objetivos de la Investigación}
			\subsection{Objetivo General}
			\begin{itemize}
				\item Proponer ...
				\item Identificar ...
			\end{itemize}
		
			\subsection{Objetivos Específicos}
			\begin{itemize}
				\item Describir ...
				\item Definir ...
				\item Conceptualizar ...
				\item Definir ...
			\end{itemize}
			
		\chapter{MARCO TEÓRICO CONCEPTUAL}	
		\section{Bases Teóricas}
	
			\subsection{Teoría ...}	
				\subsubsection{Juegos...}
				
				\begin{quote}
					Por una parte, los modelos de la teoría de juegos cooperativos están definidos por un conjunto de agentes y una función característica de beneficio o coste. Por otra parte, los modelos de la teoría de juegos no-cooperativos están definidos por un conjunto de agentes y una estructura de información que contiene un conjunto de acciones de cada agente, estrategias, pagos y la secuencia de movimientos en la interacción del juego. \parencite[p. VI]{valencia-toledoBargainingModelsAsymmetric2017} 
				\end{quote}		
				
				\subsubsection{Forma...}
					
				Especificando los tres elementos, tenemos:
				\begin{enumerate}
					\item Los jugadores.
					\item Las estrategias para cada jugador.
					\item El bienestar o Utilidad que el jugador obtiene por la unión de las estrategias posibles del juego.
				\end{enumerate}	
				
					\paragraph{Equilibrio de Nash}
					
					formalmente podríamos decir:
					
					\textit{Definición:} En un conjunto de estrategias combinadas \((s^N_1,...,s^N_n)\) es un equilibrio de Nash si para todo jugador \(i\) en \(N\) se cumple que
					\[u_i(s^N_1,...,s^N_i-1,s^N_i,s^N_i+1,...,s^N_n)\geqq u_i(s^N_1,...,s^N_i-1,s_i,s^N_i+1,...,s^N_n)\]
			
			\subsection{Colusión}
			
			Uno de los elementos centrales en las investigaciones de colusión es la dificultad de probar su existencia, ya que usualmente no se cuenta con evidencia directa como documentos escritos, grabaciones o confesiones de parte. En estos casos, las autoridades de competencia tienden a utilizar evidencia circunstancial, gran parte de la cual está conformada por evidencia económica: pruebas de la conducta de la empresa, la estructura del mercado y la existencia de prácticas que faciliten el mantenimiento de acuerdos colusorios.
		
					\paragraph{Colusión Explicita}
					
					La colusión explícita se produce cuando las empresas miembros de un oligopolio tienen comunicación y/o contacto efectivo entre ellas que les permite coordinar su comportamiento a fin de dejar de competir. En términos legales este comportamiento se califica como acuerdos o prácticas concertadas y se denomina genéricamente como carteles.
					
					\paragraph{Colusión Tácita}
					la colusión tácita supone que las empresas pueden ajustar su comportamiento al de sus rivales sin necesidad de contacto ni comunicación directa entre ellas. 
					Esto es posible debido a las particulares condiciones del mercado que les permiten detectar oportunamente la desviación del comportamiento esperado y, además, hacen creíble que aplicarán medidas de represalia frente a las desviaciones que de identifiquen. La respuesta legal frente a este comportamiento aún no se encuentra definida, existiendo planteamientos que señalan que no debería ser sancionado por ser una conducta racional y lógica en mercados con tales características.
			

\newpage
		\section{Marco Conceptual}	
		
			\subsection*{Empresa de Souvenirs}
			Agente Económico que dentro del mercado turístico se enfoca en la venta de productos recordatorios ``souvenirs'' de diversa índole, pudiendo ser productos textiles, cerámicos, orfebres, etc. 
			\subsection*{Agencias de Turismo}
			Empresa Privada que dentro del mercado turístico se encarga de la mediación, organización y realización de itinerarios para los turistas.
			\subsection*{Turista}
			Persona que se desplaza de un lugar geográfico no originario a otro con finalidad de conocer, pasear, en un lugar ajeno.
				
		\section{Antecedentes Empíricos de la Investigación}
		
		 A nivel nacional ...
		 
		
		\chapter{Hipótesis y Variables}
		\section{Hipótesis}
			\subsection{Hipótesis General}
				\begin{itemize}
				\item 
				\end{itemize}	
			\subsection{Hipótesis Especificas}
				\begin{itemize}
				\item 
				\end{itemize}	
		\section{Identificación de Variables e Indicadores}
		\section{Operacionalización de Variables}
				\input{./chapters/opera.tex}
	
		
		
		
		\chapter{Metodología}

		\section{Ámbito de estudio: localización política y geográfica}
		
		
		\section{Tipo y nivel de investigación}
			\subsection{Tipo de Investigación}
			
				
			\subsection{Nivel de Investigación}
				
				
		\section{Unidad de análisis}
			
		\section{Población de estudio}	
		
		\section{Tamaño de muestra}				
						
		\section{Técnicas de selección de muestra}
		
		\section{Técnicas de recolección de información}	
		
		\section{Técnicas de análisis e interpretación de la información}
		
		\section{Técnicas para demostrar la verdad o falsedad de las hipótesis planteadas}
		
		
		\chapter{Resultados y Discusión}
		\section{Procesamiento, análisis, interpretación y discusión de resultados}
		\section{Pruebas de hipótesis }
		\section{Presentación de resultados} %SOLO TESIS
	\input{./chapters/conclusionesrecomendaciones.tex}  %SOLO TESIS
	

{\backmatter
	
	\chapter{Presupuesto}

\begin{table}[h]
	\centering
	\begin{tabular}{|l|c|c|}
		\hline
		\textbf{RUBROS}	&	\textbf{COSTO}	&\textbf{TOTAL}	\\
		\hline
		\textbf{A)RECURSOS HUMANOS} &&\\
		-Consultor Externo	&  2000.00&\\
		-Consultor Interno	& 2000.00& \\
		-Traducción de Bibliografía& 1500.00&\\
		&&5500.00\\
		\hline
		\textbf{B)MATERIAL DE ESCRITORIO} &&\\
		-Papel Bond A4&100.00&\\
		-USB&100.00&\\
		-CD&15.00&\\
		-Plumones&25.00&\\
		-Papel Bulking&30.00&\\
		-Corrector&20.00&\\
		-Lapiceros&25.00&\\
		&&315.00\\
		\hline
		\textbf{C)SERVICIOS}&&\\
		-Movilidad&1500.00&\\
		-Viaticos&1000.00&\\
		-Tipeos&1500.00&\\
		-Empastados&600.00&\\
		-Copias&1000.00&\\
		-Anillados&450.00&\\
		-Imprevistos&1000.00&\\
		&&7050.00\\
		\hline
		&TOTAL& S/. 11365.00\\
		\hline
	\end{tabular}
	
\end{table}	%SOLO PLAN
	\chapter{Cronograma}
\newpage

\begin{table}
	\centering
	\begin{turn}{90}
	\setlength{\tabcolsep}{8pt} %espaciamiento entre columna
	\renewcommand{\arraystretch}{1.6}
	
	\begin{tabular}{|c|m{4.5cm}|c|c|c|c|c|c|c|c|c|c|c|c|c|}
		\hline
		\hspace{0.6cm}\multirow{2}{1cm}{\textbf{N}}&\hspace{0.8cm}\multirow{2}{1cm}{\textbf{ACTIVIDADES}}&\multicolumn{4}{c|}{\textbf{2018}}&\multicolumn{8}{c|}{\textbf{2019}}\\ 
		\cline{3-14} 	
		&&S&O&N&D&E&F&M&A&M&J&J&A\\
		\hline
		01&\centering Formulación del Problema&X&X&&&&&&&&&&\\
		\hline
		02&\centering Revisión Bibliográfica&&X&X&&&&&&&&&\\
		\hline
		03&\centering Elaboración del Marco Teórico&&&X&&&&&&&&&\\
		\hline
		04&\centering Elaboración del Proyecto&&&&X&X&&&&&&&\\
		\hline
		05&\centering Aprobación del Proyecto de Tesis&&&&&&&X&&&&&\\
		\hline
		06&\centering Ampliación del Marco Teórico&&&&&X&X&X&&&&&\\
		\hline
		07&\centering Programación del modelo Matemático&&&&&&&&X&X&&&\\
		\hline
		08&\centering Redacción del Informe Final&&&&&&&&&X&&&\\
		\hline
		09&\centering Aprobación del Informe Final y Sustentación&&&&&&&&&&X&&\\
		\hline							
		\end{tabular}
		\end{turn}
		
	\end{table}  %SOLO PLAN

		\chapter{Bibliografia}
	\printbibliography[heading=none] 
}


	%%%%%%%anexos%%%%%%%%%%%%%%%%%%%%
	\appendix
	\newpage
	\renewcommand{\appendixname}{Anexo}
	\chapter*{\centering Anexos}
	\addcontentsline{toc}{chapter}{Anexos}	
	\renewcommand{\thechapter}{\alph{chapter}}	%para poner numeracion en minusculas
	\thispagestyle{empty}		% para quitar numeracion       
    	\input{./chapters/matriz.tex}		
		\chapter{instrumentos de Recolección de Información}	% SOLO TESIS
		\chapter{Medios de Verificación}		% SOLO TESIS
		\input{./chapters/otros.tex}
		

	%%%%%%%%%%%%%%%%%%%%%%%%%%%%%%%%%

	
\end{document}